\chapter{Systemy kontroli wersji}
W systemach informatycznych występuje bardzo szybkie tempo zmian. Aby wspomóc programistów w zarządzaniu wersjami oprogramowania, wprowadzono systemu kontroli wersji. Najpopularniejszy obecnie jest GIT. Istnieją również platformy internetowe bazujące na tym systemie, które dostarczają interfejs graficzny i umożliwiają łatwiejszą komunikację programistom pracującym nad projektem. Najpopularniejsze platformy to Github, Gitlab czy BitBucket. Są to rozproszone systemy, dzięki temu każdy z programistów ma możliwość pracy w wydzielonym środowisku, nie wchodząc w kolizję innym programistom z zespołu.