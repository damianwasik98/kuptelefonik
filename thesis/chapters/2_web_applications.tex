\chapter{Aplikacje internetowe}

\section{Rodzaje aplikacji}
Najbardziej popularne kategorie aplikacji \cite{typy-aplikacji} to:
\begin{enumerate}
    \item aplikacje internetowe
    \item aplikacje desktopowe
    \item aplikacje mobilne
    \item aplikacje wiersza poleceń
\end{enumerate}

\subsection{Aplikacje internetowe}
Aplikacje internetowe (webowe) to opgrogramowanie dostępne z poziomu przeglądarki internetowej (takiej jak Google Chrome, Safari, czy Firefox). Przeglądarki są wbudowane w każdy popularny system operacyjny. Ich ogromną zaletą z perspektywy użytkownika jest wysoka dostępność. Żeby skorzystać aplikacji webowej, wystarczy znać jej adres internetowy. Przykładem aplikacji internetowej jest YouTube \cite{youtube}
\subsection{Aplikacje desktopowe}
Aplikacje desktopowe instaluje się bezpośrednio na komputerze. Są one dystrybuowane na każdy system operacyjny oddzielnie (np. program napisany na system Windows nie zadziała na Linux). Mogą być szybsze w działaniu, ponieważ zazwyczaj nie są uzależnione od szybkości łącza internetowego.
\subsection{Aplikacje mobilne}
Aplikacje mobilne są używane codziennie przez użytkowników smartfonów. Dzięki nim we własnej kieszeni mamy dostęp do wszystkich potrzebnych nam szybko informacji.
\subsection{Aplikacje wiersza poleceń}
Aplikacje wiersza poleceń to programy bez interfejsu graficznego. Zazwyczaj są wykorzystywane przez programistów, administratorów. Program uruchamia się komendą uruchomieniową (zazwyczaj też jest opcja podania argumentów uruchomieniowych w celu modyfikacji zachowania aplikacji). Po uruchomieniu w powłoce \footnote{(ang. shell) – program komputerowy pełniący rolę pośrednika pomiędzy systemem operacyjnym lub aplikacjami a użytkownikiem, przyjmując jego polecenia i „wyprowadzając” wyniki działania programów} widoczne są komunikaty do użytkownika świadczące o przebiegu działania programu.  

\section{W jaki sposób działa aplikacja internetowa?}

\subsection{Podział na frontend i backend}
Programowanie aplikacji internetowych dzieli się na dwie główne dziedziny:
- frontend
- backend

Zazwyczaj za każdą dziedzinę jest odpowiedzialny oddzielny zespół programisttów. Ułatwia to w ogólnym zarządzaniu w projekcie.

Frontend odpowiada za wszystko co użytkownik widzi jak się czuje korzystając z aplikacji. Za to jak strona wygląda, jakie są na niej animacje, przejścia. Jak elementy są ułożonie na stronie, jaka jest szata kolorystyczna. Programiści frontend piszą kod źródłowy serwisu w języku HTML, stylizują wygląd strony w CSS oraz odpowiadają za doświadczenie użytkownika na stronie za pomocą języka javascript.


Backend jest związany z całą logiką aplikacji. Wartości widoczne na stronie, muszą być wcześniej odpowiednio przetworzone i wyciągnięte na przykład z bazy danych. Zespoły backendowe są odpowiedzialne za takie zadania:
- komunikacja z bazami danych
- bezpieczeństwo danych
- tworzenie interfejsów API


\subsection{Architektura klient-serwer}
Aplikacje internetowe są dostępne za pomocą protokołu HTTP. Jest on zaprojektowany na bazie architektury klient-serwer. Klientem jest użytkownik, który chce wejść na stronę. Wysyła przygotowane zapytanie do serwera (serwisu, z którego chce skorzystać). Najprostszym zapytaniem jest po prostu adres internetowy strony.

\subsection{Frameworki}
Oprócz znajomości języków programowania wykorzystuje się różne frameworki. Framework to przygotowany zestaw narzędzi zaprojektowany po to, żeby ułatwiać rozwiązanie danego problemu. Popularne frameworki są utrzymywane przez zespoły programistów, zazwyczaj posiadają też obszerną dokumentacje wraz z przykładami użycia. Są też dobrze przetestowane. Dzięki temu, wszystkie sprawy związane z utrzymywaniem tego kodu są po stronie innego zespołu. Oszczędza się również dużo czasu. Zamiast myśleć nad przygotowywaniem potrzebnych narzędzi, tworzy się bezpośrednio logikę aplikacji. 

\subsection{API}
API jest to interfejs umożliwiający manipulowanie aplikacją. Jest to ustandaryzowany sposób komunikacji pomiędzy programistami. Sposobem na wykorzystanie API może być komunikacja pomiędzy zespołem frontend i backend. Zespoły backend przetwarzają dane i wystawiają tzw. końcówki (endpointy)
